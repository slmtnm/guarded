\chapter{Денотационная семантика и охраняемые команды} \label{ch1}

В этой главе рассматривается денотационный способ задания семантики 
языка программирования в терминах преобразования предикатов,
а также описывается семантика операторов языка охраняемых команд Дейкстры.

\section{Семантика языка программирования}
Построение семантики языка программирования -- формальное описание смысла, придаваемого его
элементам -- выражениям, предложениям, операторам. На настоящий момент известны
следующие способы определения семантики языка программирования:
\begin{enumerate}
	\item Операционная семантика
	\item Аксиоматическая, или деривационная семантика
	\item Денотационная семантика
\end{enumerate}

% Эти методы формального описания семантики не являются взаимоисключаищими: один и тот же
% язык може

В основе денотационной семантики является сопоставления синтаксическим единицам языка
математических объектов -- множеств и отоношений. Денотационная семантика оператора 
описывает связь между двумя состояниями: 
\begin{itemize}
	\item \textit{начальным} -- состоянием программы перед выполнением оператора
	\item \textit{конечным} -- состоянием программы после выполнения оператора
\end{itemize}

\section{Множество состояний}
Любая программа, предназначенная для выполнения на вычислительном устройстве, оперирует
с определенным множеством ячеек памяти -- записывает в них и считывает из них. Назовем это множество
ячеек памяти, или как принято называть переменных, состоянием программы. Нетрудно заметить,
что множеством всех возможных состояний программы является прямое произведение областей значений
каждой переменной. Программа может изменять состояние программы как изменением значений
имеющихся переменных, так и добавлением новых переменных. Экспоненциальный рост числа состояний 
программы с количеством переменных, также известный как проклятие размерности, дает основание 
считать множество возможных состояний практически несчетным.

Для описания подмножеств множества состояний используется 
исчисление предикатов первого порядка, в которых формулы содержат свободные термы -- 
переменные состояния. Таким образом каждому предикату соответствует некоторое множество состояний,
в интерпретации которых формула истинна.

\section{Преобразование предикатов}
Преобразователь предикатов -- это функциональное отношение на множестве предикатов.
В рамках денотационной семантики, каждой программе $S$ языка охраняемых
команд сопоставляется преобразователь предикатов $p$, который по предикату
$R$ определяет предикат $p(S, R)$.

\section{Предусловие и постусловие}
Рассмотрим некоторую программу $S$ языка охраняемых команд. Предикаты $R$ и $p(S, R)$ 
называются \textit{постусловием} и \textit{предусловием} программы $S$ соответственно, 
если программа $S$, запущенная в состоянии, удовлетворяющем $p(S,R)$, 
обязательно завершится и после своего выполнения останется в состоянии,
удовлетворяющем предикату $R$.

Предикат $wp(S, R)$ называется \textit{слабейшим предусловием}, если он 
характеризует множество \textit{всех} состояний, запуск программы $S$ из которых
обязательно приведет к завершению программы и оставит ее в состоянии,
удовлетворяющем предикату $R$.

Предикат $wlp(S, R)$ называется \textit{слабейшим свободным предусловием}, если он 
характеризует множество \textit{всех} состояний, запуск программы $S$ из которых,
если он приведет к завершению программы, оставит ее в состоянии, удовлетворяющем предикату $R$.

\section{Выводы} \label{ch1:conclusion}
Текст выводов по главе \thechapter.

Кроме названия параграфа <<выводы>> можно использовать (единообразно по всем главам) следующие подходы к именованию последних разделов с результатами по главам:
\begin{itemize}
	\item <<выводы по главе N>>, где N --- номер соответствующей главы;
	\item <<резюме>>;
	\item <<резюме по главе N>>, где N --- номер соответствующей главы.
\end{itemize}

Параграф с изложением выводов по главе \textit{является обязательным}.

%% Вспомогательные команды - Additional commands
%
%\newpage % принудительное начало с новой страницы, использовать только в конце раздела
%\clearpage % осуществляется пакетом <<placeins>> в пределах секций
%\newpage\leavevmode\thispagestyle{empty}\newpage % 100 % начало новой страницы