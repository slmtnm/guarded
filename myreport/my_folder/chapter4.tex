\chapter{Эксперименты с языком охраняемых команд} \label{ch4}

Для эксперементального вывода корректных программ, а также обратного вывода семантики программы
по ее исходному тексту, использованы примеры, приведенные Э. Дейкстрой в ??? и С.С. Лавровым в ???.

\section{Факториал числа}
Следующий алгоритм вычисляет факториал числа $N$:
\begin{verbatim}
n, f := N, 1;

{factorial(n) * f == factorial(N) & n >= 1}
do n > 1 -> f, n := f * n, n - 1 od

{f == factorial(N)}
\end{verbatim}

Программа состоит из инициализации переменных $n$ и $f$ и цикла,
умножающего $f$ на числа от 2 до $N$. Инвариант цикла

\begin{equation}
    P = \left( n! \cdot f == N! \wedge n \geq 1 \right)
\end{equation}

обеспечивает истинность постусловия по окончании этого цикла. Действительно,
после окончания работы цикла состояние программы удовлетворяет предикату $P \wedge n <= 1$. Подставляя $P$, получаем
\begin{equation}
n! \cdot f == N! \wedge n \geq 1 \wedge n \leq 1  \vdash n! \cdot f == N! \wedge n == 1 \vdash f == N!
\end{equation}

Выведенное предусловие: 
\begin{equation}
    N \geq 1
\end{equation}

Однако для функции $factorial$, введенной в постусловии необходимо показать, что
\begin{equation}
    n == 1 \wedge factorial(N) == f \cdot factorial(n) \vdash f == factorial(N)
\end{equation}

Тогда денотационная семантика прогаммы будет определена, а ее корректность -- формально доказана.

\section{Максимум двух чисел}
Следующий алгоритм вычисляет максимум из двух чисел $a$ и $b$

\section{Алгоритм Евклида}

\section{Выводы} \label{ch4:conclusion}
Текст выводов по главе \thechapter.

%% Вспомогательные команды - Additional commands
%
%\newpage % принудительное начало с новой страницы, использовать только в конце раздела
%\clearpage % осуществляется пакетом <<placeins>> в пределах секций
%\newpage\leavevmode\thispagestyle{empty}\newpage % 100 % начало новой страницы