\chapter{Описание языка} \label{ch2}
В этой главе изложено синтаксическое описание языка охраняемых команд наряду с его семантикой.
Для этого поэтапно вводятся используемые в нем операторы, понятие охраняемой команды, а в приложении
находятся полное синтаксическое описание языка с помощью расширенной формы Бэкуса-Наура (РБНФ) и 
синтаксических диаграмм Вирта.

\section{Оператор <<skip>>}
\section{Оператор <<reject>>}
\section{Оператор присваивания}
	\subsection{Множественное присваивание}
	\subsection{Литералы и типы выражений}
\section{Охраняемая команда}
\section{Условный оператор}
\section{Оператор цикла}
\section{Макроподстановки}

\section{Выводы} \label{ch2:conclusion}
Текст заключения ко второй главе. Пример ссылок \cite{Article,Book,Booklet,Conference,Inbook,Incollection,Manual,Mastersthesis,Misc,Phdthesis,Proceedings,Techreport,Unpublished,badiou:briefings}, а также ссылок с указанием страниц, на котором отображены те или иные текстово-графические объекты  \cite[с.~96]{Naidenova2017} или в виде мультицитаты на несколько источников \cites[с.~96]{Naidenova2017}[с.~46]{Ganter1999}. Часть библиографических записей носит иллюстративный характер и не имеет отношения к реальной литературе. 

Короткое имя каждого библиографического источника содержится в специальном файле \verb|my_biblio.bib|, расположенном в папке \verb|my_folder|. Там же находятся исходные данные, которые с помощью программы \texttt{Biber} и стилевого файла \texttt{Biblatex-GOST} \cite{ctan-biblatex-gost} приведены в списке использованных источников согласно ГОСТ 7.0.5-2008.
Многообразные реальные примеры исходных библиографических данных можно посмотреть по ссылке \cite{ctan-biblatex-gost-examples}.

Как правило, ВКР должна состоять из четырех глав. Оставшиеся главы можно создать по образцу первых двух и подключить с помощью команды \verb|\input| к исходному коду ВКР. Далее в приложении \ref{appendix-MikTeX-TexStudio} приведены краткие инструкции запуска исходного кода ВКР \cite{latex-miktex,latex-texstudio}.

В приложении \ref{appendix-extra-examples} приведено подключение некоторых текстово-графических объектов. Они оформляются по приведенным ранее правилам. В качестве номера структурного элемента вместо номера главы используется <<П>> с номером главы. Текстово-графические объекты из приложений не учитываются в реферате.



%% Вспомогательные команды - Additional commands
%
%\newpage % принудительное начало с новой страницы, использовать только в конце раздела
%\clearpage % осуществляется пакетом <<placeins>> в пределах секций
%\newpage\leavevmode\thispagestyle{empty}\newpage % 100 % начало новой страницы