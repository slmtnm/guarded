\chapter*{Введение} % * не проставляет номер
\addcontentsline{toc}{chapter}{Введение} % вносим в содержание
Современные программные системы становятся все более сложными, а требования к их надежности возрастают.
Процесс верификации программных систем обычно ограничивается их тестированием, то есть подтверждением их правильности
в некоторых частных случаях входных данных. Строгое формальное доказательство программ является крайне трудоемким,
поэтому выполняется только в критических местах программного обеспечения, цена ошибки в котором слишком велика. Примером
такого программного обеспечения является прошивка бортовых компьютеров ракет и самолетов, военные оборонные системы, 
прошивка медицинского оборудования. 

Наиболее распространенными методами формальной верификации программ являются \textit{проверка моделей} и доказательство теорем в некотором, соответствующем
исследуемому языку \textit{исчислении программ}. Проверка моделей пригодна для использования, когда программная система моделируется
несколькими взаимодействующими конечными автоматами, и определяет ее соответствие спецификации -- набору формул в темпоральной логике.
Существует множество программных реализаций для автоматической проверки моделей: SPIN, NuSMV, BLAST, TLA+. Некоторые из них предлагают
специальный язык для построения моделей, а остальные выполняют статический анализ программ для уже имеющихся языков программирования общего назначения.
Второй подход основан на аксиоматической семантике языка: для того, чтобы доказать корректность программы, требуется вывести из аксиом формулу специального вида 
для этой программы. Аксиомы при этом отражают семантику операторов языка. Как и в случае с проверкой моделей, существуют системы для автоматического
вывода формул в исчислении программ: Coq, Idris, PVS.

Э. Дейкстра в своей работе <<Дисциплина программировния>> \cite{Dijkstra} рассматривает другой подход к созданию программ,
в котором доказательство их корректности является одним из главных этапов их построения. Таким образом, корректность программы устанавливается
не после построения, а в процессе. С этой целью он предлагает язык программирования и снабжает
его множеством примеров.

При построении сложных алгоритмов используется подход, предложенный Н.Виртом в статье <<Program Development by Stepwise Refinement>>\cite{Wirth}.
В методе пошагового уточнения весь процесс создания программы состоит из нескольких уточняющих шагов. 
Программа представляет собой набор инструкций, и на каждом шаге уточнения выбирается одна из них и декомпозируется в несколько
более детальных инструкций. Весь процесс продолжается до тех пор, пока каждая инструкция не будет являться оператором языка программирования
или инструкцией вычислительной машины. 

% ОБЪЕКТ ИССЛЕДОВАНИЯ
Объектом исследования работы является практическая реализация интерпретатора языка и
инструмента для статической верификации кода программ, а также проверка реализованного языка на примерах построения корректных программ. 

% ПРЕДМЕТ ИССЛЕДОВАНИЯ
Предметом исследования работы является реализация языка охраняемых команд Дейкстры и инструмента для построения
денотационной семантики программ с помощью заданных инвариантов в программном коде. Язык охраняемых команд 
должен содержать средства, позволяющие демонстрировать процесс построения программ на этом языке методом пошагового уточнения.

% Автоматический синтез корректных программ не является предметом исследования работы, 
% однако статический анализ программ, полуавтоматически выводящий их семантику, может служить полезным и удобным инструментом для формальной верификации. 

% ЦЕЛЬ ИССЛЕДОВАНИЯ
Целью исследований является проверка разработанного языка охраняемых команд на конкретных примерах программ.

Для достижения этой цели необходимо решить следующие задачи:
\begin{enumerate}
	\item описать язык формально -- определить грамматику языка и все его операторы
	\item разработать интерпретатор программ на языке охраняемых команд, исполняющий ее и выводящий результат
	\item разработать программу-анализатор, позволяющей по исходному тексту программы выводить предусловие, если постусловие и инварианты заданы пользователем
	\item провести эксперементы разработанных инструментов на примерах программ
\end{enumerate} 


% \textbf{Теоретическая и методологическая база исследования}. 
Теоретической основой работы послужили исследования Э. Дейкстры 
в своей работе <<Дисциплина программирования>>\cite{Dijkstra}, дополненные результатами, полученными С.С. Лавровым в книге <<Математические основы, средства, теория>>\cite{Lavrov},
а также предложенный Н. Виртом\cite{Wirth} метод пошагового уточнения.
Практическая часть работы выполнялась на основании примеров, предложенных Э. Дейкстрой.