\chapter*{Заключение} \label{ch-conclusion}
\addcontentsline{toc}{chapter}{Заключение}	% в оглавление 

В ходе данной работы реализован язык охраняемых команд Дейкстры:
\begin{enumerate}
	\item Описана денотационная семантика языка охраняемых команд путем сопоставления каждым базовым операторам языка своего
	преобразователя предикатов
	\item Разработан интерпретатор языка охраняемых команд, позволяющий исполнять программы и выводить конечное состояние
	в виде набора переменных и их значений 
	\item Реализован вывод денотационной семантики программы с 
помощью заданных пользователем инвариантов и желаемого постусловия программы. Результатом вывода является предусловие, предъявляемое
к начальному состоянию программы, и истинность которого гарантирует истинность постусловия в конечном состоянии программы. Для линейно
ветвлящихся программ, не включающих операторы цикла, вывод предусловия происходит в автоматическом режиме, так как
не требует указания инвариантов. Инварианты указываются в программном коде рядом с циклами, к которым они относятся. Постусловие указывается
в конце кода программы.
	\item Рассмотрены несколько примеров программ на языке охраняемых команд, корректность которых проверена с помощью разработанных инструментов.
\end{enumerate}

Язык охраняемых команд является удобным средством для реализации подхода создания программ, корректных по построению,
который описан Э. Дейкстрой в работе <<Дисциплина программирования>>. Данный подход рассмотрен как альтернатива
другим способам формальной верификации программ, таким как проверка моделей или автоматическое доказательство теорем,
однако он может использоваться и совместно с ними. Такой подход значительно увеличивает время разработки программ
и выдвигает требование к квалификации программиста, поэтому он не может использоваться повсеместно. Разработанный 
инструментарий может быть вспомогательным средством для реализации этого подхода.

% Заключение (2 -- 5 страниц) обязательно содержит выводы по теме работы, \textit{конкретные
% предложения и рекомендации} по исследуемым вопросам. Количество общих выводов
% должно вытекать из количества задач, сформулированных во введении выпускной
% квалификационной работы.

% Предложения и рекомендации должны быть органически увязаны с выводами
% и направлены на улучшение функционирования исследуемого объекта. При разработке
% предложений и рекомендаций обращается внимание на их обоснованность,
% реальность и практическую приемлемость.

% Заключение не должно содержать новой информации, положений, выводов и
% т. д., которые до этого не рассматривались в выпускной квалификационной работе.
% Рекомендуется писать заключение в виде тезисов.